%%
%
% ARQUIVO: pre-texto.tex
%
% VERSÃO: 1.0
% DATA: Maio de 2016
% AUTOR: Coordenação de Trabalhos Especiais SE/8
% 
%  Arquivo tex para a criação da parte pré-textual do documento de Projeto de Fim de Curso.
%
%%


% -----
% PÁGINA DE CAPA DO DOCUMENTO DE PFC
% -----
\makecapa

% -----
% PÁGINA DE TÍTULO DO PFC
% -----
\prepareadvisors
\maketitle

% -----
% PÁGINA DE CRÉDITOS DO DOCUMENTO DE PFC
% -----

% -----
% PÁGINA DE FOLHA DE ASSINATURAS
% -----
\preparemembers

% -----
% PÁGINA DE DEDICATÓRIA (OPCIONAL, ie. pode remover toda a página)
% -----
%%% DEDICATÓRIA - PREENCHER...

% -----
% PÁGINA DE AGRADECIMENTOS (OPCIONAL, ie. pode remover toda a página)
% -----
%%% AGRADECIMENTOS - PREENCHER...
\agradecimentos{%
Agradeço ao Professor Roberto, por ter me mostrado o ramo de estudos sobre as linguagens de programação.  \\
\indent
Agradeço ao Professor Ivan, por ter me ensinado a codificar, que tanto tentei aprender sozinho sem sucesso.\\
\indent
Agradeço especialmente aos desenvolvedores do Valgrind, que possibilitaram a escrita deste software. \\
\indent
Agradeço especialmente aos contribuidores da esolangs.org, que escreveram os códigos da página \textit{Brainfuck Algorithms} sem os quais esse projeto seria impossível.\\
\\
\indent
Agradeço aos meus professores da PUC por terem fornecido o ambiente no qual eu pude desenvolver tais conhecimentos. 
}%
\makethanks

% -----
% PÁGINA DE EPÍGRAFE (OPCIONAL, ie. pode remover toda a página)
% -----
%%% EPÍGRAFE - PREENCHER...
\epigrafe{%
Os poetas não enlouquecem, mas os jogadores de xadrez, sim.
}%
\autorepigrafe{%    %% Se não tem autor, coloque "Anônimo"
Alan J. Perlis
}%
\epigrafe{%
I think that it's extraordinarily important that we in computer science keep fun in computing. When it started out, it was an awful lot of fun..
}%
\autorepigrafe{%    %% Se não tem autor, coloque "Anônimo"
Alan J. Perlis
}%
\makeepigraph

% -----
% PÁGINA DE SUMÁRIO
% -----
\tableofcontents

% -----
% PÁGINAS DE LISTAS DE FIGURAS E DE TABELAS
% se a Dissertação não possui figuras e/ou tabelas, REMOVA O COMANDO CORRESPONDENTE
% -----


% -----
% PÁGINA DE LISTA DE SIGLAS
% se a Dissertação não possui siglas, REMOVA TODA A PÁGINA
% -----
%%% SIGLAS - PREENCHER...


% -----
% PÁGINA DE LISTA DE ABREVIATURAS
% se a Dissertação não possui abreviaturas ou símbolos, REMOVA TODA A PÁGINA
% -----
%%% ABREVIATURAS - PREENCHER...


% -----
% PÁGINA DE RESUMO
% -----
%%% RESUMO - PREENCHER...
\resumo{%
Menezes Lucas. Ierusalimschy Roberto, Compilando pra Brainfuck – A Linguagem Headache - 44 páginas - Pontifícia Universidade Católica do Rio de Janeiro.
Este documento descreve a linguagem de programação Headache e o seu compilador, hac (Headache Compiler).  Headache é uma linguagem de programação estruturada (bastante parecida com C) que é compilada para brainfuck de 8 bits. Neste documento é discutido como foi implementado, quais foram as técnicas, o modelo de memória e as estratégias utilizadas para fazer o sistema funcionar.\\
\\
Palavras-chaves: Brainfuck, Headache, Compilador, Linguagem de Programação.
}%
\makeresumo

% -----
% PÁGINA DE ABSTRACT
% -----
%%% ABSTRACT - PREENCHER...
\abstract{%
Menezes Lucas. Ierusalimschy Roberto, Compiling to Brainfuck – The Headache Programming Language -  44 pages - Pontifícia Universidade Católica do Rio de Janeiro
This document describes the Headache Programming Language and its compiler, hac (Headache Compiler). Headache is a structured programming Language (which is very C-like) that is compiled to 8 bit brainfuck. In this document we discuss how it was implemented, the structures that were used, its memory model and the strategy used to bring everything working together.\\
\\
Keywords: Brainfuck, Headache, Compiler, Programming Language.
}%
\makeabstract
