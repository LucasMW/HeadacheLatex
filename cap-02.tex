%%
%
% ARQUIVO: cap-02.tex
%
% VERSÃO: 1.0
% DATA: Maio de 2017
% AUTOR: Carla Cosenza, Matheus Mello, Rebeca Reis
% 
%  Arquivo tex de exemplo de capítulo do documento de Projeto de Fim de Curso.
%
% ---
% DETALHES
%  a. todo capítulo deve começar com \chapter{•}
%  b. usar comando \noindent logo após \chapter{•}
%  c. citações para referências podem ser
%       i. \citet{•} para citações diretas (p. ex. 'Segundo Autor (2015)...'
%       ii. \citep{•} para citações indiretas (p. ex. '... (AUTOR, 2015)...'
%  d. notas de rodapé devem usar dois comandos
%       i. \footnotemark para indicar a marca da nota no texto
%       ii. \footnotetext{•}, na sequência, para indicar o texto da nota de rodapé
%  e. figuras devem seguir o exemplo
%       i. devem ficar no diretório /img e devem ser no formato EPS
%  f. tabelas devem seguir o exemplo
%  g. figuras e tabelas podem ser colocadas em orientação landscape
%       i. figuras: usar \begin{sidewaysfigure} ... \end{sidewaysfigure}
%                   em vez de \begin{figure} ... \end{figure}
%       ii. tabelas: usar \begin{sidewaystable} ... \end{sidewaystable}
%                    em vez de \begin{table} ... \end{table}
%  h. toda figura e tabela deve ser referenciada ao longo do texto com \ref{•}
% ---
%%

\chapter{Situação Atual}
\noindent
Escrever programas em \textit{Brainfuck} é muito difícil. Isso faz com que a biblioteca de exemplos em \textit{Brainfuck} seja muito curta.

Escrever, no entanto, programas de exemplo em linguagens estruturadas é fácil. Já existem algumas soluções que compilam de uma linguagem estruturada para \textit{Brainfuck}, porém são desconhecidas, a maioria dela são apenas soluções pessoais.

\section{Descrição e avaliação de tecnologias e sistemas existentes}
Em 2017, durante o desenvolvimento de Headache, surgiu uma solução similar nomeada \textit{Brain} feita em \textit{Rust}. Ela utiliza a sintaxe inspirada em \textit{Rust} e compila para \textit{Brainfuck}. Headache se diferencia da brain por usar uma sintaxe inspirada em C  (e em monga, por extensão). Além disso, o compilador hac foca em ser uma implementação leve (os binários pesam em torno de 100KB ) e rápida de um compilador, inclusive no tempo de compilação (3 segundos). 

Já a Brain, demora 474.3 segundos para compilar no release,  possui 1,6 MB e mais 12 MB de dependência. 
Existem outras soluções mais antigas como a ebf feitas nos anos 2000, porém estão perdidas em pastas compactadas em páginas antigas da internet. Muitas vezes não tem documentação, ou simplesmente não tem nenhuma usabilidade.

Headache propõe contrapor esse ponto. A usabilidade hac é baseada diretamente na utilização do gcc e do clang. Algumas features como modo interativo foram adicionadas pensando no comportamento das distribuições linguagens interpretadas,.

A usabilidade do repositório é baseada diretamente em padrões de simplicidade do github e de projetos unix. O repositório foi feito justamente para ser clonado, compilado e estar usável em menos de um minuto.
