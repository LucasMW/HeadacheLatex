%%
%
% ARQUIVO: cap-02.tex
%
% VERSÃO: 1.0
% DATA: Maio de 2017
% AUTOR: Carla Cosenza, Matheus Mello, Rebeca Reis
% 
%  Arquivo tex de exemplo de capítulo do documento de Projeto de Fim de Curso.
%
% ---
% DETALHES
%  a. todo capítulo deve começar com \chapter{•}
%  b. usar comando \noindent logo após \chapter{•}
%  c. citações para referências podem ser
%       i. \citet{•} para citações diretas (p. ex. 'Segundo Autor (2015)...'
%       ii. \citep{•} para citações indiretas (p. ex. '... (AUTOR, 2015)...'
%  d. notas de rodapé devem usar dois comandos
%       i. \footnotemark para indicar a marca da nota no texto
%       ii. \footnotetext{•}, na sequência, para indicar o texto da nota de rodapé
%  e. figuras devem seguir o exemplo
%       i. devem ficar no diretório /img e devem ser no formato EPS
%  f. tabelas devem seguir o exemplo
%  g. figuras e tabelas podem ser colocadas em orientação landscape
%       i. figuras: usar \begin{sidewaysfigure} ... \end{sidewaysfigure}
%                   em vez de \begin{figure} ... \end{figure}
%       ii. tabelas: usar \begin{sidewaystable} ... \end{sidewaystable}
%                    em vez de \begin{table} ... \end{table}
%  h. toda figura e tabela deve ser referenciada ao longo do texto com \ref{•}
% ---
%%

\chapter{Considerações finais}
\noindent

\section{Contribuições desse trabalho para a comunidade}
Usuários e criadores linguagens esotéricas se animaram com o trabalho. O repositório de Headache teve um fork do criador da brain-lang (uma extensão de brainfuck com mais comandos, não confundir com o projeto homônimo similar à Headache)  para adaptar o hac para que este compilasse para brain. Headache também teve uma entrada (escrita por mim) na esolangs.org. 

Conforme foi mencionado anteriormente, a "pesquisa em brainfuck” continua ocorrendo: A Brainfuck Algorithms recebe novos códigos; otimizadores de brainfuck são criados; projetos similares a Headache surgiram no mesmo ano, mesmo sem contato comum conhecido; e outras linguagem esotéricas baseadas em brainfuck continuam a ser criadas.

Headache coroa como uma dessas pesquisas, com atualmente 14 estrelas no github. (Sexto lugar na pesquisa “esoteric-programming-language” ; Décimo-quinto na pesquisa “esolang” e décimo terceiro na tag “brainfuck” - quando o relatório foi escrito).

Se Headache alcançar 50 estrelas no github, pode ser considerado famoso o suficiente para figurar no Homebrew: o package manager não oficial do mac OS (segundo o critério do Homebrew para criar novos pacotes) .

\section{O que foi aprendido}
Melhor domínio do git. Manusear forks públicos no github. Extensão de bitwidth em brainfuck. Uso de elipses em C. Otimização básica de constantes. Redigir make install. Redigir interpretação de parâmetro; Redigir testes que envolvessem a execução de de código gerado. Utilizar o minGW para compilar programas C Unix no Windows.

Se eu começasse a fazer Headache hoje, pensaria em implementá-la em rust contra implementá-lo em C. Mas o mais importante é que eu faria o source do zero ao invés de começar a trabalhar a partir do source do mongaComp, apenas utilizando-o para consulta. Por mais que tenha me permitido refatorar o código do projeto original, isso acabou acoplando código desnecessário ao hac que acabou passando às revisões. O código poderia ser muito reduzido se isso não tivesse sido feito. 

Também o faria direto numa implementação sem registradores, baseada apenas no uso de heap fixo e liberado após cada execução, tudo calculado estaticamente (no momento de compilação).


\section{Oportunidades para trabalhos futuros}
Um próximo aluno poderia resolver os bugs do hac ou implementar mais features. Headache abre espaços para receber muitas contribuições de tipos diferentes: 

Implementar novas features na linguagem, por exemplo: 

Arrays dinâmicos; 

Variáveis real; 

(Existe uma lista de features planejadas que não chegaram a ser implementadas neste mesmo relatório). Seriam todas extremamente bem vindas a integrar o repositório.

Implementar novas otimizações em -O1 -O2 ou -O3.

Otimizar o código do hac. 

Nota: Há uma diferença entre otimizar os algoritmos e o modelo de memória que Headache se baseia para gerar o código output e adicionar otimizações de AST (que é o que é ativado nas opções -O1, -O2, -O3). As otimizações de AST são, basicamente, conversões de conjuntos de instruções em conjuntos menores e mais rápidos que tenham o mesmo efeito ainda dentro da representação interna do compilador (Headache -> Headache). É diferente de otimizar a geração de código, isto é, como certas estruturas da linguagem são transformadas em código output.

Fazer um fork no repositório github para que Headache seja compilada em uma variante de brainfuck, ou outra linguagem esotérica como whitespace.

Headache foi concebida para ser open source e receber pull requests de vários colaboradores. Há muitos ramos diferentes que podem ser explorados por uma aluno que queira realizar melhorias

\phantom{\cite{Peres}, \cite{Pegg}, \cite{Clifford}, \cite{Hughes}, \cite{Github}, \cite{Libbf}, \cite{BrainfuckCode}, \cite{BrainfuckUtil}, \cite{GoogleCode}, \cite{BFO}, \cite{BrainfuckDummies}, \cite{BrainfuckA}, \cite{Brainfuck}}
